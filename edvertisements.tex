\documentclass{sigchi}

% Use this command to override the default ACM copyright statement
% (e.g. for preprints).  Consult the conference website for the
% camera-ready copyright statement.


%% EXAMPLE BEGIN -- HOW TO OVERRIDE THE DEFAULT COPYRIGHT STRIP -- (July 22, 2013 - Paul Baumann)
% \toappear{Permission to make digital or hard copies of all or part of this work for personal or classroom use is      granted without fee provided that copies are not made or distributed for profit or commercial advantage and that copies bear this notice and the full citation on the first page. Copyrights for components of this work owned by others than ACM must be honored. Abstracting with credit is permitted. To copy otherwise, or republish, to post on servers or to redistribute to lists, requires prior specific permission and/or a fee. Request permissions from permissions@acm.org. \\
% {\emph{CHI'14}}, April 26--May 1, 2014, Toronto, Canada. \\
% Copyright \copyright~2014 ACM ISBN/14/04...\$15.00. \\
% DOI string from ACM form confirmation}
%% EXAMPLE END -- HOW TO OVERRIDE THE DEFAULT COPYRIGHT STRIP -- (July 22, 2013 - Paul Baumann)


% Arabic page numbers for submission.  Remove this line to eliminate
% page numbers for the camera ready copy 

%\pagenumbering{arabic}

% Load basic packages
\usepackage{balance}  % to better equalize the last page
\usepackage{graphics} % for EPS, load graphicx instead 
%\usepackage[T1]{fontenc}
\usepackage{txfonts}
\usepackage{times}    % comment if you want LaTeX's default font
\usepackage[pdftex]{hyperref}
% \usepackage{url}      % llt: nicely formatted URLs
\usepackage{color}
\usepackage{textcomp}
\usepackage{booktabs}
\usepackage{ccicons}
\usepackage{todonotes}

\usepackage{paralist}

% llt: Define a global style for URLs, rather that the default one
\makeatletter
\def\url@leostyle{%
  \@ifundefined{selectfont}{\def\UrlFont{\sf}}{\def\UrlFont{\small\bf\ttfamily}}}
\makeatother
\urlstyle{leo}

% To make various LaTeX processors do the right thing with page size.
\def\pprw{8.5in}
\def\pprh{11in}
\special{papersize=\pprw,\pprh}
\setlength{\paperwidth}{\pprw}
\setlength{\paperheight}{\pprh}
\setlength{\pdfpagewidth}{\pprw}
\setlength{\pdfpageheight}{\pprh}

% Make sure hyperref comes last of your loaded packages, to give it a
% fighting chance of not being over-written, since its job is to
% redefine many LaTeX commands.
\definecolor{linkColor}{RGB}{6,125,233}
\hypersetup{%
  pdftitle={Edvertisements: Adding Microlearning to Social News Feeds and Websites},
  pdfauthor={LaTeX},
  pdfkeywords={SIGCHI, proceedings, archival format},
  bookmarksnumbered,
  pdfstartview={FitH},
  colorlinks,
  citecolor=black,
  filecolor=black,
  linkcolor=black,
  urlcolor=linkColor,
  breaklinks=true,
}

% create a shortcut to typeset table headings
% \newcommand\tabhead[1]{\small\textbf{#1}}

% End of preamble. Here it comes the document.
\begin{document}

\title{Edvertisements: Adding Microlearning to \\ Social News Feeds and Websites}

\numberofauthors{3}
\author{%
  \alignauthor{1st Author Name\\
    \affaddr{Affiliation}\\
    \affaddr{City, Country}\\
    \email{e-mail address}}\\
  \alignauthor{2nd Author Name\\
    \affaddr{Affiliation}\\
    \affaddr{City, Country}\\
    \email{e-mail address}}\\
  \alignauthor{3rd Author Name\\
    \affaddr{Affiliation}\\
    \affaddr{City, Country}\\
    \email{e-mail address}}\\
}

\maketitle

%\begin{textblock}{5}(3.6,7.7)
%\begin{figure}
%\includegraphics[width=\columnwidth]{feedlearn-screenshot.png}
%\caption{FeedLearn showing an interactive vocabulary quiz inside a user's Facebook news feed}
%\label{fig:feedlearn}
%\end{figure}
%\end{textblock}

\begin{figure}
\includegraphics[width=\columnwidth]{feedlearn-screenshot.png}
\caption{Our extension can show interactive microlearning tasks (Edvertisements) in users' Facebook news feeds.}
\label{fig:feedlearn}
\end{figure}

\begin{figure}
\includegraphics[width=\columnwidth]{edvertisements-screenshot4.png}
\caption{Our extension can replace advertisements with interactive microlearning tasks (Edvertisements) on arbitrary websites.}
\label{fig:edvertisements}
\end{figure}

\begin{abstract}
Many long-term goals, such as learning a language, require people to spend a small amount of time each day to achieve them. At the same time, people regularly surf the web and read social news feeds in their spare time. We have built a browser extension that teaches vocabulary in the context of Facebook feeds and arbitrary websites, by showing users interactive quizzes they can answer without leaving the website. On Facebook, the quizzes are shown as part of the news feed, while on other sites, the quizzes are shown where advertisements would normally appear. In our user study, we looked at the effectiveness of inserting microlearning tasks into social news feeds. We compared vocabulary learning rates when interactive quizzes were inserted directly into feeds, versus inserting links that lead them to quizzes. Our results suggest that users engage with and learn from our inserted quizzes, and engagement is higher when they can be done directly inside their feeds.
%Many long-term goals, such as learning a language, require people to spend a small amount of time each day to achieve them. At the same time, people regularly surf the web and read social news feeds in their spare time. We have built a browser extension that teaches vocabulary in the context of Facebook feeds and arbitrary websites, by showing users interactive quizzes they can answer without leaving the website. On Facebook, the quizzes are shown as part of the news feed, while on other sites, the quizzes are shown where advertisements would normally appear. In our preliminary user study, we looked at the effectiveness of inserting microlearning tasks into social news feeds. We compared Japanese vocabulary learning rates when interactive quizzes were inserted directly into feeds, versus inserting links that lead them to quizzes. Our results suggest that users engage with and learn vocabulary from our inserted quizzes, and they engage more with microlearning tasks when they can be done directly inside their feeds.
% In our preliminary user study, we find that over the course of a week, we are able to teach students an average of 13 new vocabulary words by injecting microlearning tasks into their Facebook feeds. This is more than the 4 vocabulary words learned if we insert links leading them to visit an external site to study, as is done by current Facebook apps.
% Unlike apps like Duolingo posting accomplishments in users' feeds, FeedLearn gives friends a simple call to action and allows them to participate in their friends' learning processes.
%In our first study, we find that over the course of 2 weeks, we are able to teach students N new vocabulary words by injecting microlearning tasks into their Facebook feeds. This is X\% more than a control condition where we ask them to visit a separate website and ask them to study the words on their own time, and Y\% more than a control condition where we show friends' achievements via feed updates. In our second study, we find that giving friends the ability to participate in the social learning process, by suggesting words for their friends to learn, leads to increased engagement. They learn more words in a condition where the words learned are suggested to the group by their friends, which is X\% more than when they select words to choose by themselves, and Y\% more than when the suggestions are automatic.
\end{abstract}

\keywords{microlearning; social feeds; facebook; language learning}

\category{H.5.m.}{Information Interfaces and Presentation (e.g. HCI)}{Miscellaneous}

%\newpage

\section{Introduction}

People spend large amounts of time surfing the web and reading social news feeds on sites like Facebook.
American adults spend an average of 27 hours per month browsing the web \cite{nielsen2014}.
71\% of American adults with an internet connection use Facebook. Of these, 63\% visit Facebook at least once a day, and 40\% visit it multiple times per day \cite{socialmediaupdate}. Among American college students, 90\% use Facebook \cite{collegefacebook2}, spending an average of 30 minutes per day on it \cite{collegefacebook}. Social news feeds are widely used - over half of college students who use Facebook report reading their Facebook news feeds 5-7 days per week \cite{collegefacebook}. % Social news feeds and % Clearly, Facebook news feeds present an opportunity for influencing the behavior of users.

In this paper, we present Edvertisements, interactive microlearning tasks which we show to users as they browse the web and read their Facebook feeds. We implemented a Chrome extension which shows Edvertisements in two ways:

\begin{compactitem}
\item On Facebook, Edvertisements are inserted into the feed, alongside regular feed items.
\item On other sites, Edvertisements are shown in locations where advertisements would normally appear.
\end{compactitem}

\pagebreak

Our research questions are:

\begin{compactitem}
\item Do users engage and learn from Edvertisements that we insert into their Facebook feeds?
\item Do users engage and learn more with Edvertisements if they can do the microlearning tasks without leaving their Facebook feeds (compared to external links)?
% \item Does having Edvertisements where users can do the microlearning tasks without leaving their Facebook feeds result in higher learning outcomes?
% \item Does the regularity and frequency with which users visit Facebook make it suitable for microlearning?
%\item Are people more likely to learn if they are shown small, easily actionable learning tasks in the social feed. %, instead of messages emphasizing friends' overall achievements?
%\item  Are people more likely to do flashcards that their friends suggested for them to do, as opposed to flashcards they selected themselves, or flashcards suggested by the system?
%\item  Are people more likely to do flashcards that they suggested to their friends for them to do, as opposed to flashcards they selected only for themselves, or flashcards suggested by the system?
%\item Does the act of suggesting words for friends to learn improve users' engagement with the system, or perceived satisfaction?
%\item Does the selection process result in something useful?
\end{compactitem}

In our user study, we looked at engagement and vocabulary acquisition when Edvertisments were inserted into users' Facebook feeds. We also compared the effects on engagement and vocabulary acquisition of being able to do the do the microlearning tasks in place, as opposed to having to click a link to go to an external website. We found that there was high engagement with Edvertisements, users had improved post-test results after a week, and that engagement was higher when the Edvertisements could be done without leaving the feed. % users answered more quizzes when they could do so without leaving the feed, and they learned more new words on average over a week. 

% Our preliminary user study looked at engagement and Japanese vocabulary acquisition rates through FeedLearn's in-feed interactive quizzes, versus inserting links to an external website where they can do quizzes, as is currently done by Facebook applications. We found that users answered more quizzes when they could do so without leaving the feed, and they learned more new words on average over a week. % when they could do the quizzes inside the feeds.

\section{Related Work}

\subsection{Microlearning}

Microlearning is a strategy of using short periods of time throughout the day to study. It has been used for applications such foreign vocabulary learning via mobile apps \cite{microlearning} \cite{micromandarin}. A potential drawback of needing a separate app for microlearning is that it requires the user to develop a habit of interrupting their routine to open an app to study. %vAlthough they may give users reminders when they should practice, these reminders might come at the wrong time, or be ignored.

Some systems have attempted to solve this problem by embedding microlearning into other contexts. There are games where users complete learning tasks while playing \cite{carriearcade}, video players which teach vocabulary while watching foreign-language videos \cite{smartsubtitles}, screensavers that show facts while the screen is idle \cite{screensaver}, and chat clients that show vocabulary while the user is chatting \cite{cai2015wait}.

Compared to the learning contexts used by existing work, we believe that recreational web surfing and Facebook feeds are especially good opportunities for microlearning, because:

\begin{compactitem}
\item Unlike playing educational games or watching foreign-language videos, visiting Facebook is part of the daily routine of nearly half of American adults with an internet connection \cite{socialmediaupdate} %, and the majority of college students \cite{collegefacebook}.
% \item Unlike a screensaver which is dismissed once the mouse moves, users can interact with quizzes they see in their Facebook feeds.
%\item Unlike needing to respond to a chat message, there are no interruptions to the user's learning while they are browsing their Facebook feeds.
\item Web surfing and reading Facebook news feeds are recreational activities, so the inserted microlearning tasks will not interrupt users' work.
\item Users are already used to a variety of rich content appearing in their Facebook feeds, such as videos, games, recommendations, and advertisements.
%\item Users are already used to a variety of rich content appearing in their Facebook feeds, such as videos their friends liked, posts from games and apps, recommendations, and advertisements.
\end{compactitem}


%\subsection{Spaced Repetition}

%Spaced repetition is a technique designed to help learners retain information by having them review items at regular intervals \cite{karpicke2011spaced}. A class of applications that exploit this are flashcards, which split information into independent chunks that are scheduled for review based on factors such as mastery and recency of review. There have been a number of algorithms and models designed for optimizing learners' retention of the material via spaced repetition \cite{optimalschedule} \cite{memreflex}.

\subsection{News Feeds as a Persuasive Technology}

Many apps attempt to use Facebook feeds as a persuasive technology. For example, apps like Duolingo can broadcast users' study progress on the platform, inviting the user's friends to participate in the activity. However, there are many caveats with such applications auto-posting messages on users' feeds. Messages auto-posted by applications receive less attention from the user's friends, compared to messages posted by actual users. Viewers may perceive these posts negatively, ignoring them \cite{socialsharing}.

\pagebreak

\subsection{Web Advertising and Ad-Blocking}

Although advertisements are an important revenue source for websites, in consumer surveys 77\% report that they hardly ever click on ads, and 69\% express interest in skipping or blocking ads \cite{adblockinggames}. Ad blockers, which are browser extensions that prevent web ads from being displayed, are used by 5\% of all internet users \cite{adblockinggoesmainstream}. Ad-blocking is especially common among Chrome and Firefox users -- 30\% of Chrome users, and 35\% of Firefox users, have installed an ad-blocker \cite{adblockinggoesmainstream}. %with ad-blocking extensions being the most popular extensions in each browser's extension store.

In surveys, users of ad-blockers cite ``distracting animations and sounds'', and ``offensive/inappropriate ad content'' as their top reasons for blocking ads \cite{adblockinggames}. Although ad-blocking software can be configured to selectively whitelist/blacklist ads for certain websites, most ad-blocker users just retain the default behavior of blocking ads on all websites \cite{adblockinggames}, which may perhaps be due to usability issues in configuring ad-blockers \cite{adblockusability}. By preventing ads from being shown, ad-blockers pose a threat to the advertising industry, as well as websites which rely on advertising revenue \cite{adblockinggames}. % Some websites such as Hulu have responded to this by detecting the presence of ad-blocking software, and requesting that users disable their ad-blockers, though writers of ad-blocker.

% Many apps attempt to use Facebook feeds as a persuasive technology. For example, apps like Duolingo can broadcast users' study progress on the platform, inviting the user's friends to participate in the activity. %A key advantage that social platforms like Facebook provide is that friends can be associated with requests, increasing their potential persuasiveness via social pressures \cite{foggfacebook}.
%Since the emergence of the Facebook app development platform, there have been
%many attempts to use it as a platform for persuasion. For example, apps like NikePlus can broadcast users' running progress, and apps like Duolingo can broadcast users' study progress on the platform. These messages may also invite the user's friends to participate in the activity. A key advantage that social platforms like Facebook provide is that friends can be associated with requests, increasing their potential persuasiveness via social pressures \cite{foggfacebook}.

% However, there are many caveats with such applications auto-posting messages on users' feeds. Messages auto-posted by applications receive little attention from the user's friends, compared to messages that they have posted themselves. Viewers may perceive these posts as either trivial achievements or bragging, ignoring them \cite{socialsharing}. % It is thus suggested that these auto-posted messages be shared only with the subset of the user's friends who are likely to be interested. However, users are unwilling to invest the effort to identify these social circles \cite{socialsharing}.

%\subsection{Study groups on Facebook}

%There are a number of Facebook pages that post daily ``word of the day'' style lessons for learners, such as KoreanClass101. If users subscribe to these pages (by clicking the Like button), they will see periodic reminders to visit an external site to study vocabulary. These services have a number of weaknesses that FeedLearn aims to address:

% There are a number of Facebook pages that post daily ``word of the day'' style lessons for learners, such as KoreanClass101. If users subscribe to these pages (by clicking the Like button), they will see periodic reminders to visit an external site to study vocabulary, as shown in \autoref{fig:learn-korean}. These services have a number of weaknesses that FeedLearn aims to address:

%\marginpar{
%\begin{figure}
%\includegraphics[width=\marginparwidth]{learn-korean-post.png}
%\caption{An example Daily Word post from KoreanClass101, a Facebook service with 70 thousand subscribers.}
%\label{fig:learn-korean}
%\end{figure}
%}

%\begin{compactitem}
%\item Not interactive: users need to visit an external site to do quizzes or see other words.
%\item Not personalized: all 70 thousand subscribers will see the same daily word posted, regardless of whether they already know that word.
%\item No spaced repetition: a new word is posted each day, and older words are never repeated.
%\item Content needs to be manually generated: a group moderator needs to write a new post each day
%\end{compactitem}

% because . These services also generally rely on an external site for 

% ALOE is a system that allows users to learn foreign-language vocabulary while browsing the web, by replacing words in the user's native language with foreign-language vocabulary \cite{augmenting}. This work differs from existing microlearning systems by 

\section{Edvertisements System}

Our system is a Chrome extension that inserts microlearning tasks -- in our case, vocabulary quizzes -- into users' Facebook feeds, and as they are browsing the web. Although we originally implemented the browser extension for Chrome, we have also ported it to Firefox, and our technique can be implemented on any browser that supports extensions (Chrome, Firefox, Edge, Safari, etc). Our system has a variety of microlearning tasks for learning vocabulary in multiple languages, but in this paper we will focus on learning Japanese vocabulary.

% IAB standard container

% Internet Advertising Board

\subsection{Inserting Edvertisements into Facebook Feeds}

Our extension can insert Edvertisements into users' Facebook feeds, as rectangular interactive quizzes mimicking the look of a regular feed item, as shown in \autoref{fig:feedlearn}. We chose to insert 1 microlearning task for every 10 normal feed items, to mimic the approximate frequency we observed sponsored content appearing in the feed.

\subsection{Replacing Web Advertisements with Edvertisements}

People spend considerable time on sites other than Facebook, so we also wished to have a general mechanism for presenting microlearning tasks as users browse the web. We do so by detecting web advertisements on pages, and replacing them with microlearning tasks.

We detect the presence of web advertisements the same way ad blockers do -- by checking the URL the element is loaded from, and comparing it against EasyList, which is a public list of known URL patterns for advertisements maintained by Adblock Plus. If the element is detected as an advertisement, we then replace it with a quiz of the same size.

% There are standardized sizes for web advertisements, called the IAB (Interactive Advertising Bureau) Standard Ad Units, which over 80\% of online advertisements follow.
Web advertisements follow standardized sizes, called the IAB (Interactive Advertising Bureau) Standard Ad Units.
We have implemented microlearning tasks to fit 2 of the common sizes -- 300x250 and 200x90 -- which correspond to regular-sized and small ads.  If a microlearning task in the appropriate size is not available, we pick a smaller one and scale and stack it to fit the available space. For example we can fill a banner ad (728x90) with 3 small Edvertisements, as shown in \autoref{fig:edvertisements}. %Our system chooses the

% citation for the 80% figure: A guide to IAB's new standard ad units http://www.imediaconnection.com/content/31499.asp 

%\section{}

%\section{FeedLearn Interface}

%FeedLearn inserts interactive vocabulary quizzes into users' Facebook feeds, as shown in \autoref{fig:feedlearn}. It is implemented as a Chrome extension, as Facebook's API does not currently allow developers to insert interactive content into feeds. FeedLearn supports multiple languages, but this paper will focus on learning basic Japanese nouns. % FeedLearn supports multiple languages, but this paper will focus on the version that teaches basic Japanese nouns.

\subsection{Quiz Types}

One type of quiz presents a noun in English, and asks the user to select the corresponding Japanese word, as shown in \autoref{fig:quiz1}. To ensure that users learn word associations in both ways, we also have a second type of quiz, where the user is shown a word in Japanese and selects the corresponding word in English, as shown in \autoref{fig:quiz2}.
% To ensure that users learn word associations in both ways, we also have a second type of quiz, where the user is shown a word in Japanese and selects the corresponding word in English, as shown in \autoref{fig:quiz2}.

\begin{figure}
\centering
\includegraphics[width=1.0\columnwidth]{quiz1}
\caption{One type of quiz presents a noun in Japanese (\textit{jikan}), and asks the user to select its meaning (time).}
\label{fig:quiz1}
\end{figure}

\begin{figure}
\centering
\includegraphics[width=1.0\columnwidth]{quiz2}
\caption{Another type of quiz presents a noun in English (umbrella), and asks the user to select the correct translation into Japanese (\textit{kasa}). The user has incorrectly selected \textit{fukuro}, so the user is shown its meaning (bag), and tries again.}
\label{fig:quiz2}
\end{figure}

We opted to use this multiple-choice quiz format, because it tests the user's knowledge with a minimal amount of interaction -- the user simply clicks on a word to answer. Once the user answers a quiz correctly, a new quiz testing a different word is shown. Thus, users can use an edvertisement to continue study vocabulary for as long as they wish to.

% A picture is also presented alongside the foreign-language word, to help learners visually remember them. Prior research has shown that flashcards showing a picture along with the text allow learners to learn vocabulary better than text-only flashcards \cite{multimediavocabulary}. We focus on nouns, because they are the most common type of word -- the majority of words in the Oxford English dictionary are nouns \cite{microlearning} -- and they are relatively easy to visualize. This is the same quiz style used by Duolingo to introduce nouns.

% We opt to use this interactive quiz format, rather than simply showing pairs of words and translations or asking users to explicitly recall and type out translations for words, because it allows us to take advantage of the testing effect with a minimal amount of interaction -- the user simply clicks on a word to answer. Once the user answers a quiz correctly, a new quiz testing a different word is shown. Thus, users can continue to study in their feed for as long as they wish to.

 \subsection{Quiz Generation}

Our words and definitions were taken from the Nouns section of Wiktionary's 1000 Basic Japanese Words list. We excluded loanwords that users would easily recognize (\textit{pinku}=pink), and words that are homographs when romanized (\textit{hana}=flower or nose). We focus on nouns, because they are the most common type of word \cite{microlearning}. % -- the majority of words in the Oxford English dictionary are nouns % FeedLearn can optionally show the word in the native script (kana/kanji for Japanese), or display a picture of the word. However, we did not show these in our user study, because our users could not read Japanese scripts, and not all words can be visualized with a picture (ex: year).

\subsection{Spaced Repetition}

Spaced repetition algorithms schedule items for review to ensure long-term retention \cite{karpicke2011spaced}. We modified the Memreflex algorithm \cite{memreflex} to show the word due for review that has been seen least recently in the feed, as opposed to always showing the most overdue word as Memreflex does. This ensures that users will continue to see different words as they are scrolling through their feeds, even if they are not always answering the in-feed questions.

% Spaced repetition algorithms schedule items for review to ensure long-term retention \cite{spacedrepetition}. We modified the Memreflex algorithm \cite{memreflex} to show the word due for review that has been seen least recently in the feed, as opposed to always showing the most overdue word as Memreflex does. This ensures that users will continue to see different words as they are scrolling through their feeds, even if they are not always answering the in-feed questions.

% Quizzes are generated automatically from the provided English word. The other options are generated by looking up the word in WordNet \cite{wordnet}, to find related words that have similar category and semantics but have different meaning (for example, other animal names). The images are obtained by taking the first result on Google Images. We select the list of words to teach based on their overall usefulness in the language (as judged by word frequency in natural-language text).

%\subsection{Inserting Quizzes into Feeds}

%We insert quizzes into feeds so that they will be encountered at a rate of 1 quiz every 10 posts. We picked this rate, as this was similar to the frequency we observed advertisements and sponsored content appearing in Facebook news feeds. Hence, our should not distract users any more than existing advertisements. %on Facebook. % This is optimal because (need justification).

\section{User Study}

We conducted a preliminary user study to see how much users would engage with Edvertisements, and compare the effectiveness of inserting interactive quizzes that can be done without leaving the page, versus inserting static links as is done by today's web advertisements and sponsored Facebook posts. %compare the effectiveness of inserting interactive quizzes directly into users' Facebook feeds, versus inserting links to the quizzes as is commonly done on Facebook today. % We compared the 

% have a figure illustrating that usage did not decrease over itme

In our study, we only inserted microlearning tasks into Facebook feeds and did not replace ads with quizzes, as Facebook feeds were an environment we could better control the frequency with which quizzes appeared, their size and appearance, and many interested users were already using ad-blockers, which would have conflicted with the ad-replacement functionality.

\subsection{Participants}

We recruited 12 users who had not previously studied Japanese but were interested in learning some basic vocabulary.  5 were female, 7 male.They were voluntary participants recruited from online forums and Facebook groups related to Japanese culture. All of our participants self-reported that they were regular users of Facebook. %, spending at least 10 minutes on the site each day.

% We conducted a 2-week between-subjects user study to evaluate the effectiveness of our system in teaching foreign-language vocabulary.

% We recruited 30 college students with no prior exposure to Japanese, who were interested in learning some basic vocabulary in Japanese. All of our participants were regular users of Facebook, spending at least 10 minutes on the site each day.

\begin{figure}
\centering
\includegraphics[width=1.0\columnwidth]{feedlearn-link-screenshot}
\caption{The control condition in our user study inserted a link into users' Facebook feeds that led them to a site where they could do vocabulary quizzes}
\label{fig:control}
\end{figure}

\subsection{Materials}

We used 50 basic Japanese words from Wiktionary's Basic Japanese Words list as the study material. We presented vocabulary words in romanized form instead of Japanese script, as our users could not read Japanese script.

%We selected the 30 most frequently used non-abstract nouns in Japanese, and generated flashcards automatically for them, according to the procedure described in the Quiz Generation section. We presented vocabulary words in romanized form rather than the standard Japanese orthography, to avoid difficulties resulting from unfamiliarity with the Japanese writing system.

\subsection{Conditions}

Users were assigned to one of two conditions:

\begin{compactitem}
\item Users in the \textit{in-feed quiz} condition had quizzes inserted directly in their feeds, as shown in \autoref{fig:feedlearn}.
\item Users in the \textit{link} condition had links inserted into their feed which led them to a site where they could do the quizzes, as shown in \autoref{fig:control}.
\end{compactitem}

Apart form the different items (quizzes/links) inserted into the feed, the questions and quiz interfaces were identical in the two conditions. In both conditions, the items were inserted at a rate of 1 quiz/link per 10 feed items. % We chose this rate because it was approximately the rate at which we observed sponsored content and advertisements to appear in our feeds.

% There were 3 conditions in our study. We assigned 10 students to each:

\subsection{Procedure}

The study was conducted entirely online. First, users took a pre-test on the words we were intending to teach them, where they tried matching the 50 Japanese words to their 50 English definitions. Then they installed our Chrome extension and used it to study the 50 words for a week. After a week, we asked them users to do the post-test, which had the same format as the pre-test.

\section{Results}

\subsection{Vocabulary Quiz Results}

\begin{figure}
\centering
\includegraphics[width=1.0\columnwidth]{vocab-test-scores}
\caption{Vocabulary test scores for the in-feed quiz and link conditions, with standard error bars}
\label{fig:vocab-test-scores}
\end{figure}

Average vocabulary pre-test and post-test scores are shown in \autoref{fig:vocab-test-scores}. On average, users in the in-feed condition learned 13.2 new words, compared to 2.5 new words learned in the link condition. However, this was not statistically significant (t=1.51, p=0.16).

\subsection{Engagement With Edvertisements}

The number of times users practiced answering quizzes is shown in \autoref{fig:event-logs}. We also kept track of ``study sessions'', which we defined as the number of times the user clicked on the link to visit the external website (in the link condition), or first answered a quiz that was inserted into their feed (in the in-quiz condition). % We also show number of answers and sessions normalized by the number of feed insertions to normalize for amounts of Facebook use.

\begin{figure}
\centering
\includegraphics[width=1.0\columnwidth]{event-logs-feedlearn}
\caption{Average number of events logged per user for the in-feed quiz and link conditions.}
\label{fig:event-logs}
\end{figure}

We found that there was high engagement with in-feed Edvertisements -- on average, users answered 116 quizzes across 21 study sessions and answered a question 4.4 days of the 5.7 days they visited Facebook. Users in the in-feed quiz condition answered significantly more quizzes than the link condition, did more study sessions, and studied on more days across the week. % We believe this difference is due to the decreased barrier to starting a study session in the in-feed condition, as they do not need to leave the feed.

\subsection{Qualitative Feedback}

Some users mentioned that they would prefer words to be explicitly introduced first before they start appearing in quizzes.  In addition, as shown by our ``ratio of study sessions to insertions'', even in the in-feed condition, users only interact with 1/4 of quizzes that they see. Hence, we need to ensure that seeing items reinforces memory, even if users do not interact with them. To address this issue, we later added new types of items to introduces new words and review old ones.

\section{Conclusion and Future Work}

Edvertisements are interactive microlearning tasks that we can show to users as they are surfing the web. We have built a browser extension that shows Edvertisements by inserting them into Facebook feeds, and by replacing web advertisements with them.

In our user study, we inserted Edvertisements teaching vocabulary into users' Facebook feeds. We found that there was users engaged with and learned from Edvertisements, and that engagement rates were higher when the quizzes could be done without leaving their feeds.

Our current implementation inserts Edvertisements into Facebook feeds, and replaces web advertisements with them. There are other online recreational contexts where we might show users Edvertisements -- for example, between youtube videos, chapters of a e-book, or in their email.

Although we have focused on microlearning vocabulary, Edvertisements might also be a vehicle for reminding people to do other small, beneficial tasks as they are idly surfing the web -- for example, encouraging people to brush their teeth, do a small exercise, or complete an item on their to-do list.

Another direction for future work includes making the Edvertisements contextually relevant to the page they are being inserted into. For example, if it is your Chinese friend's birthday, your Facebook feed might show an Edvertisement teaching you how to wish him happy birthday in Chinese. Or if you are reading an anti-vaccination webpage, an Edvertisement might teach you scientific facts about vaccines.

Like the problem posed by ad-blockers today, an issue with replacing web advertisements with Edvertisements is that websites will be unable to monetize their content. Monetization is another area of future work. One potential approach would be sponsored Edvertisements -- for example, a local gym might sponsor an Edvertisement which teaches you a workout routine, hoping it'll encourage you to visit their gym more. % Alternatively, one might investigate effects of having a mix of Edvertisements and normal ads. It would be interesting if, for example, the presence of Edvertisements starts making people pay more attention to web advertising in general.

% Just like the problem posed by ad-blockers today, an issue with entirely replacing web advertisements with Edvertisements is that websites will be unable to monetize their content. Making this system acceptable and appealing to advertisers is another area of future work. One potential approach would be sponsored Edvertisements -- for example, a local gym might sponsor an Edvertisement which teaches you an workout routine, hoping it'll encourage you to visit their gym more. Alternatively, one might investigate effects of replacing only some of the ads, and having a mix of Edvertisements and traditional advertisements. It would be interesting if, for example, the presence of Edvertisements starts making people pay more attention to web advertising in general.

\section{Edvertisements Demo and Source Code}

Researchers interested in using or building Edvertisements can see \url{http://edvertisements.github.io/}

\pagebreak

% FeedLearn uses Facebook feeds for vocabulary microlearning.
% By eliminating the need to leave the Facebook feed to do quizzes, FeedLearn reduces the barrier
% required to start microlearning tasks. Our user study found that eliminating the need to click a link to start studying vocabulary results in increased engagement.

% We have thus far focused on interactive microlearning tasks in -- Facebook feeds, and replacing web advertisements. There are many other contexts where we might show users relevant microlearning tasks -- for example, between youtube videos.

% One potentially interesting 

% Future work includes making the microlearning tasks contextually relevant -- for example, if the user is reading articles about 



% Although we have focused on microlearning vocabulary, other content could also be conveyed in the context of social feeds and advertisements.  In-feed messages encouraging small, actionable tasks could also be used to promote habits such as microexercise.

% Future work includes using a model to determine the optimal times to insert microlearning tasks into feeds. Another potential extension is making the microlearning tasks more integrated with the Facebook environment to create a more social in-feed learning experience. %Future work includes better integrating the microlearning tasks with the the social nature of the Facebook environment to create a richer in-feed learning experience.

%\section{Acknowledgements}

%Thanks to James Landay, Michael Bernstein, and the Stanford HCI group for suggestions on this project. Geza Kovacs is supported by the NDSEG fellowship.

%Although we have focused on vocabulary learning, this approach can potentially be applied to other educational content, or even be used to encourage other small, actionable behaviors such as microexercise. % One could also use a similar mechanism to encourage other behaviors that need to be done for short periods of time on a daily basis, such as small amounts of exercise.

%Potential future work includes using an adaptive model to determine the optimal times to insert microlearning tasks into feeds. Another potential extension of this work is making the microlearning tasks more integrated with the Facebook environment to create a more social in-feed learning experience.

% The study is conducted entirely online. First, we ask the users to take a pre-test, to verify that they do not already know the vocabulary that we intend to teach them. The test is a multiple choice test: there are 30 questions total, with 15 questions where questions where the user is given the foreign-language word and is asked to select the correct English translation, and 15 questions where the user is given a word in English and is asked to select the correct foreign translation.

% Then, we have them install our extension have them use the service for 2 weeks (according to the condition they have been assigned to). After the 2 weeks have elapsed, we again ask them to take the vocabulary quiz.

%\subsection{Research Questions}

%H1: Do users learn more vocabulary in the FeedLearn condition than the other conditions?

%H2: Do user engage more (complete more quizzes) in the FeedLearn condition than the other conditions?

%H3: Do users enjoy the FeedLearn condition more than the other conditions?

%\section{Results}

%H1: Hopefully, we find that the amount of vocabulary learned in the FeedLearn condition is greater than the other two conditions

%H2: Hopefully, we find the users engage more (complete more quizzes) in the FeedLearn condition than the other conditions.

%H3: Hopefully, we find that users find injection of quizzes directly into their Facebook feeds less distracting than injection of reminders for them to visit the site, or email-based reminders.

% \section{Discussion}

% Although in all 3 conditions, users receive daily reminders to go study vocabulary, we found that users complete more quizzes in the FeedLearn condition, and consequently learn more vocabulary.

% Relative to the External Service with In-Feed Reminders condition, we can attribute this to the reduced friction required for the interaction: the user no longer needs to leave their news feed and visit an external service to review vocabulary, so the barrier to engagement is lowered by inserting vocabulary quizzes directly into social feeds.

% Relative to the External Service with Email Reminders condition, we can additionally attribute an advantage to being able to catch them when they are idle and free. While the user may be busy at 10AM and not have time when they receive the email to review the vocabulary, and will consequently have to remember to go visit the site at a later time, users in the FeedLearn condition have already indicated that they are free and idle because they are browsing their news feed, and are hence more likely to be available to review vocabulary when they encounter them in their feeds.

% Users found the FeedLearn condition more enjoyable and less distracting than the External Service with Email Remdiners condition, because they were able to directly go engage with the quiz content, and were not distracted by unnecessary reminder emails.

% REFERENCES FORMAT
% References must be the same font size as other body text.
\bibliographystyle{SIGCHI-Reference-Format}
\bibliography{edvertisements}

\end{document}

%%% Local Variables:
%%% mode: latex
%%% TeX-master: t
%%% End:

